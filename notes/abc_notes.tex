\documentclass[aps,pra,onecolumn,showpacs,floatfix]{revtex4}
%\usepackage{graphics}
%\usepackage[utf8x]{inputenc}
\usepackage{bm}
\usepackage{graphicx} % for PDF output
\usepackage{amsmath}
\usepackage{braket}
%\usepackage{subfigure}
\begin{document}
\title{Reactive Scattering Notes.}
\author{James F. E. Croft}
\affiliation{\noaffiliation}

\date{\today}

\begin{abstract}
Reactive scattering notes.
\end{abstract}

\pacs{}

\maketitle
%
\section{Introduction}
\section{Jacobi Coordinates}
A, B and C with mass $m_\tau$ ($\tau$ = A,B,C) and \textbf{x}$_\tau$ being the
column vectors of their coordinates relative to an origin fixed in the
laboratory. After the separation of the center of mass motion, the Jacobi
coordinates for relative motion are
\begin{align}\label{eqn:jac}
	\mathbf{R}_\tau = \mathbf{x}_
	\tau-\frac{m_{\tau+1} \mathbf{x}_{\tau+1}+m_{\tau+2} \mathbf{x}_{\tau+2}}{m_{\tau+1} + m_{\tau+2}} \\
	\mathbf{r}_\tau = \mathbf{x}_{\tau+2} - \mathbf{x}_{\tau+1}
\end{align}
where $\tau$, $\tau+1$ and $\tau+2$ are any cyclic permutation of A,B and C. The
corresponding mass-scaled Jacobi coordinates are then
\begin{align}\label{eqn:ms_jac}
	\mathbf{S}_\tau = d_\tau\mathbf{R}_\tau \\
	\mathbf{s}_\tau = d^{-1}_\tau\mathbf{r}_\tau
\end{align}
where
\begin{align}\label{eqn:ms_jac}
d_\tau = \left[\frac{m_\tau}{\mu}\left(1-\frac{m_\tau}{M}\right)\right] \\
M = m_A + m_B + m_C
\end{align}
The advantage of using mass scaled Jacobi coordinates is the kinetic energy
operator has the simple form
\begin{align}\label{eqn:ms_jac}
	T = -\frac{\hbar}{2\mu}(\nabla^2_{\mathbf{S}_\tau} + \nabla^2_{\mathbf{s}_\tau})
\end{align}
in body fixed coordinates
\begin{align}\label{eqn:ms_jac}
	T = -\frac{\hbar}{2\mu}\left[\frac{1}{S_\tau}\frac{\partial^2}{\partial S^2_\tau}S_\tau + \frac{1}{s_\tau}\frac{\partial^2}{\partial s^2_\tau}s_\tau\right] +
	\frac{\mathbf{L}^2_\tau}{2\mu S^2_\tau} + \frac{\mathbf{J}^2_\tau}{2\mu S^2_\tau}
\end{align}
where $\mathbf{J_\tau}$ is the angular momentum of the diatom  and
$\mathbf{L_\tau}$ the orbital angular momentum of the atom $\tau$ around the
diatom. The total angular momentum $\mathbf{J}=\mathbf{J_\tau}+\mathbf{L_\tau}$

\section{Delves Hyperspherical Coordinates}
Delves coordinates\cite{Delves:1958,Delves:1960}
\begin{align}\label{eqn:delves}
	\rho = (\mathbf{S}^2_\tau + \mathbf{s}^2_\tau ) \\
	\theta_\tau = \tan^{-1}({\frac{s_\tau}{S_\tau}})
\end{align}
the hyperradius $\rho$ is independent of arrangement channel $\tau$.

\section{Asymptotic Basis in jacobi coordinates}
Constructs an asymptotic multiple-arrangement ro-vibrational basis set.
Asymptotically $\lim\limits_{S\to\infty}$

\section{Sbasis}
\begin{acknowledgments}
\end{acknowledgments}

\bibliography{notes}

\end{document}
